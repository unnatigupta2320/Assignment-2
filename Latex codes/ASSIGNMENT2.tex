\documentclass[journal,12pt,twocolumn]{IEEEtran}

\usepackage{setspace}
\usepackage{gensymb}

\singlespacing


\usepackage[cmex10]{amsmath}

\usepackage{amsthm}

\usepackage{mathrsfs}
\usepackage{txfonts}
\usepackage{stfloats}
\usepackage{bm}
\usepackage{cite}
\usepackage{cases}
\usepackage{subfig}

\usepackage{longtable}
\usepackage{multirow}

\usepackage{enumitem}
\usepackage{mathtools}
\usepackage{steinmetz}
\usepackage{tikz}
\usepackage{circuitikz}
\usepackage{verbatim}
\usepackage{tfrupee}
\usepackage[breaklinks=true]{hyperref}
\usepackage{graphicx}
\usepackage{tkz-euclide}

\usetikzlibrary{calc,math}
\usepackage{listings}
    \usepackage{color}                                            %%
    \usepackage{array}                                            %%
    \usepackage{longtable}                                        %%
    \usepackage{calc}                                             %%
    \usepackage{multirow}                                         %%
    \usepackage{hhline}                                           %%
    \usepackage{ifthen}                                           %%
    \usepackage{lscape}     
\usepackage{multicol}
\usepackage{chngcntr}

\DeclareMathOperator*{\Res}{Res}

\renewcommand\thesection{\arabic{section}}
\renewcommand\thesubsection{\thesection.\arabic{subsection}}
\renewcommand\thesubsubsection{\thesubsection.\arabic{subsubsection}}

\renewcommand\thesectiondis{\arabic{section}}
\renewcommand\thesubsectiondis{\thesectiondis.\arabic{subsection}}
\renewcommand\thesubsubsectiondis{\thesubsectiondis.\arabic{subsubsection}}


\hyphenation{op-tical net-works semi-conduc-tor}
\def\inputGnumericTable{}                                 %%

\lstset{
%language=C,
frame=single, 
breaklines=true,
columns=fullflexible
}
\begin{document}


\newtheorem{theorem}{Theorem}[section]
\newtheorem{problem}{Problem}
\newtheorem{proposition}{Proposition}[section]
\newtheorem{lemma}{Lemma}[section]
\newtheorem{corollary}[theorem]{Corollary}
\newtheorem{example}{Example}[section]
\newtheorem{definition}[problem]{Definition}

\newcommand{\BEQA}{\begin{eqnarray}}
\newcommand{\EEQA}{\end{eqnarray}}
\newcommand{\define}{\stackrel{\triangle}{=}}
\bibliographystyle{IEEEtran}
\providecommand{\mbf}{\mathbf}
\providecommand{\pr}[1]{\ensuremath{\Pr\left(#1\right)}}
\providecommand{\qfunc}[1]{\ensuremath{Q\left(#1\right)}}
\providecommand{\sbrak}[1]{\ensuremath{{}\left[#1\right]}}
\providecommand{\lsbrak}[1]{\ensuremath{{}\left[#1\right.}}
\providecommand{\rsbrak}[1]{\ensuremath{{}\left.#1\right]}}
\providecommand{\brak}[1]{\ensuremath{\left(#1\right)}}
\providecommand{\lbrak}[1]{\ensuremath{\left(#1\right.}}
\providecommand{\rbrak}[1]{\ensuremath{\left.#1\right)}}
\providecommand{\cbrak}[1]{\ensuremath{\left\{#1\right\}}}
\providecommand{\lcbrak}[1]{\ensuremath{\left\{#1\right.}}
\providecommand{\rcbrak}[1]{\ensuremath{\left.#1\right\}}}
\theoremstyle{remark}
\newtheorem{rem}{Remark}
\newcommand{\sgn}{\mathop{\mathrm{sgn}}}
\providecommand{\abs}[1]{\vert#1\vert}
\providecommand{\res}[1]{\Res\displaylimits_{#1}} 
\providecommand{\norm}[1]{\Vert#1\rVert}
%\providecommand{\norm}[1]{\lVert#1\rVert}
\providecommand{\mtx}[1]{\mathbf{#1}}
\providecommand{\mean}[1]{E[ #1 ]}
\providecommand{\fourier}{\overset{\mathcal{F}}{ \rightleftharpoons}}
%\providecommand{\hilbert}{\overset{\mathcal{H}}{ \rightleftharpoons}}
\providecommand{\system}{\overset{\mathcal{H}}{ \longleftrightarrow}}
	%\newcommand{\solution}[2]{\textbf{Solution:}{#1}}
\newcommand{\solution}{\noindent \textbf{Solution: }}
\newcommand{\cosec}{\,\text{cosec}\,}
\providecommand{\dec}[2]{\ensuremath{\overset{#1}{\underset{#2}{\gtrless}}}}
\newcommand{\myvec}[1]{\ensuremath{\begin{pmatrix}#1\end{pmatrix}}}
\newcommand{\mydet}[1]{\ensuremath{\begin{vmatrix}#1\end{vmatrix}}}
\numberwithin{equation}{subsection}
\makeatletter
\@addtoreset{figure}{problem}
\makeatother
\let\StandardTheFigure\thefigure
\let\vec\mathbf
\renewcommand{\thefigure}{\theproblem}
\def\putbox#1#2#3{\makebox[0in][l]{\makebox[#1][l]{}\raisebox{\baselineskip}[0in][0in]{\raisebox{#2}[0in][0in]{#3}}}}
     \def\rightbox#1{\makebox[0in][r]{#1}}
     \def\centbox#1{\makebox[0in]{#1}}
     \def\topbox#1{\raisebox{-\baselineskip}[0in][0in]{#1}}
     \def\midbox#1{\raisebox{-0.5\baselineskip}[0in][0in]{#1}}
\vspace{3cm}
\title{ASSIGNMENT-2}
\author{UNNATI GUPTA}
\maketitle
\newpage
\bigskip
\renewcommand{\thefigure}{\theenumi}
\renewcommand{\thetable}{\theenumi}
Download all python codes from 
\begin{lstlisting}
https://github.com/unnatigupta2320/Assignment-2/tree/master/codes
\end{lstlisting}
%
and latex-tikz codes from 
%
\begin{lstlisting}
https://github.com/unnatigupta2320/Assignment-2/tree/master
\end{lstlisting}
%
\section{Question No. 2.36}
Construct a quadrilateral MIST where $MI = 3.5, IS = 6.5, \angle M = 75 \degree, \angle I = 105 \degree$ and $\angle S = 120 \degree$.
%
\section{SOLUTION}

\begin{enumerate}
\item Let us assume vertices of given quadrilateral $MIST$ as $\vec{M}$,$\vec{I}$,$\vec{S}$ and $\vec{T}$.

\item Let us generalize the given data:
    \begin{align}
    &\angle M= 75\degree=\theta \label{eq1}
    \\
    &\angle I= 105\degree=\alpha
    \\
    &\angle S= 120\degree=\gamma \label{eq2}
    \\
    &\norm{\vec{I}-\vec{M}} =3.5=a, \label{eq3}
    \\
    &\norm{\vec{S}-\vec{I}} =6.5=b,\label{eq4}
    \\
    &\vec{M}=\myvec{0\\0}, \vec{I}=\myvec{3.5\\0}
    \end{align}
\item Also, Let us assume the other two sides as
\begin{align}
 &\norm{\vec{S}-\vec{T}}=c \label{eq5}
 \\
  &\norm{\vec{M}-\vec{T}} =d \label{eq6}
\end{align}  
\item Finding out that quadrilateral is possible or not:-
\begin{itemize}
\item For this quadrilateral $MIST$ we have,
\begin{align}
\angle M +\angle I = 75\degree + 105\degree =180\degree,
\end{align}
$ \implies MT \parallel IS (\because \text {MI being the transversal})$
\\
\item As, sum of adjacent angle on same side is $180\degree$ only when lines are \textbf{parallel}.Also,
\begin{align}
\implies \angle S + \angle T = 180\degree 
\\
\implies \angle T = 60\degree 
\\
\text{Let }\angle T = 60\degree =\beta \label{eq7}
\end{align}
\item Now sum of all the angles given and \eqref{eq7} is $360\degree$.So construction of given quadrilateral is \textbf{Possible}.
 \end{itemize}
 \item For finding coordinates of S:-
\begin{itemize}
\item Using vector equation of line,we have:
\begin{align}
\vec{S}=\vec{I} + \lambda m \label{eq8}
\\
\norm{\vec{S}-\vec{I}} = \abs{\lambda}\times\norm{\myvec{\cos \theta\\\sin \theta}} 
\\
\implies \norm{\vec{S}-\vec{I}}=\abs{\lambda}
\end{align}
\item Now using \eqref{eq4} and putting its value in above equation,we get
\begin{align}
\implies \abs{\lambda}=b 
\\
\implies \lambda =6.5
\end{align}
\item Now by putting value of $\lambda$ in \eqref{eq8}  and using \eqref{eq1} solving we get,
\begin{align}
&\implies \vec{S}=\myvec{I} +b\myvec{\cos\theta \\\sin \theta }
\\
&\implies \vec{S}=\myvec{3.5\\0} +6.5\myvec{\cos 75 \degree\\\sin 75 \degree} 
\\
&\implies \vec{S}=\myvec{5.18\\6.27}
\end{align}
\end{itemize}
\item For finding coordinates of T:-
\begin{itemize}
\item Again,using vector equation of line,we have:
\begin{align}
\vec{T}=\vec{M} + \mu m =\mu m (\because \vec{M}=0) 
\\
\vec{T}=\mu \myvec{\cos \beta\\\sin \beta } \label{eq9}
\end{align}
\item Using inner products of vectors we get,
\begin{align}
\frac{(S-T)^\intercal(S-I)}{\norm{S-T}\times\norm{S-I}}=\cos\gamma \label{eq10}
\\
\frac{(S-T)^\intercal(M-T)}{\norm{S-T}\times\norm{M-T}}=\cos \beta \label{eq11}
\end{align}
\end{itemize}
\item Now, dividing \eqref{eq10} and \eqref{eq11} we get:
\begin{align}
\frac{(S-T)^\intercal(S-I)}{(S-T)^\intercal(M-T)} \times \frac{\norm{M-T}}{\norm{S-I}}=\frac{\cos \gamma}{\cos \beta} 
\\
\frac{(S-T)^\intercal(S-I)}{(S-T)^\intercal(M-T)} \times \frac{\cos \beta}{\cos \gamma}= \frac{\norm{S-I}}{\norm{M-T}}
\end{align}
\item Now using \eqref{eq4}, \eqref{eq5} and \eqref{eq6} we get
\begin{align}
\frac{c^Tb}{c^T d} \times \frac{\cos \beta}{\cos \gamma} =\frac{b}{d}
\end{align}
\begin{itemize}
    \item On solving we get:
\end{itemize}
\begin{align}
\implies 6.491 \mu^2 -48.012\mu =6.4935\mu^2-48.035\mu
\end{align}
\begin{align}
&\implies \mu=9.35
\end{align}
\item Now,putting value of $\mu$ in \eqref{eq9}we have \textbf{T} as
\begin{align}
\implies \vec{T}=\abs{\mu} \myvec{\cos \beta\\\sin \beta }
\\
\implies \vec{T} =9.35\myvec{\cos 75 \degree\\\sin 75 \degree} 
\\
\implies \vec{T}=\myvec{2.42\\9.63} 
\end{align}
   \item Now,the vertices of given Quadrilateral MIST can be written as,
\begin{align}
 \vec{M}=\myvec{0\\0},\vec{I} = \myvec{3.5\\0}, \vec{S}=\myvec{5.18\\6.27},\vec{T}=\myvec{2.42\\9.63}
\end{align}
    \item On constructing the quadrilateral $MIST$ we get:
\end{enumerate}
\numberwithin{figure}{section}
\begin{figure}[!ht]
\centering
\includegraphics[ width=\columnwidth, height=12 cm]{MIST-Quadrilateral.png}
\caption{Quadrilateral MIST}
\label{fig:Quadrilateral MIST}	
\end{figure}
\end{document}